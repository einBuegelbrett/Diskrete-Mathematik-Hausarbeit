% !TeX program = xelatex
% ↑ Automatische Auswahl für XeLaTeX compiler

% Das ist mein Template für die TX000 Arbeiten. Nicht perfekt, also falls ihr Verbesserungsvorschläge habt, stellt gerne einen Pull-Request. https://github.com/NikomitK/TX000_Template
% Bitte lasst auch einen star auf github da, danke.
% Wenn ihr die cite funktion von LaTeX nutzen wollt, müsst ihr einfach die Quellen im bibtex Format in die sources.bib datei kopieren, google scholar z.B. hat bei den Quellen einen Button mit dem man das so bekommt, auch viele andere websites.
% Bilder kommen in den images Ordner, den müsst ihr beim abrufen eines Bildes nicht angeben, passiert automatisch.


% Trag hier deine Daten ein, die entsprechenden Felder werden automatisch angepasst.
\def\meinTitel{Das Rucksackproblem}
\def\artDerArbeit{HAUSARBEIT}
\def\meinName{Alexander Regemann, Sven Sendke}
\def\meinKurs{TINF22F}
\def\meineMatrikelNr{4296627 \& 8469950}
\def\modul{Diskrete Mathematik}
\def\abteilungsName{Abteilungsname}
\def\projektBetreuer{Vorname Nachname}
\def\dozent{Dipl.-Math. Christian Kratochwil}
\def\abgabeDatum{15.07.2024}
%-----------------------------------------------------------------------------------


\documentclass[12pt]{report}
\usepackage[heightrounded]{geometry}
\geometry{
	a4paper,
	lmargin=2.5cm, %Seitenrand left
	tmargin=2.5cm, %Seitenrad top
	headsep=35pt %Abstand von Kopfzeile
}
\usepackage[onehalfspacing]{setspace}
\usepackage[compact]{titlesec}
\usepackage{cite} % Zitierungen
\usepackage{struktex} % Struktogramme
\usepackage{array} % kp hat chatgpt benutzt
\usepackage{longtable} % Tabelle über einen seitenumbruch
\usepackage[nohyperlinks, printonlyused]{acronym} % abkürzungsverzeichnis
\usepackage{fontspec}
\usepackage{blindtext} %LoremIpsum
\usepackage{fancyhdr} %Kopf- und Fußzeile
\usepackage[export]{adjustbox} %Bilder alignment
\usepackage{stfloats} %Tabular at bottom
\usepackage{amsmath}

\usepackage{blindtext}

\usepackage[utf8]{inputenc} % this is needed for umlauts
\usepackage[ngerman]{babel} % this is needed for umlauts
\usepackage[T1]{fontenc}    % this is needed for correct output of umlauts in pdf


\usepackage{subcaption} % Für subfigures glaub


\usepackage{tikz} % Zum zeichnen
\usetikzlibrary{calc}
\usetikzlibrary{shapes.geometric, arrows}
\setmainfont{Arial}

%--------------------Flowcharts--------------------
\tikzstyle{startstop} = [rectangle, rounded corners, minimum width=3cm, minimum height=1cm,text centered, draw=black, fill=red!30]
\tikzstyle{io} = [trapezium, trapezium left angle=70, trapezium right angle=110, minimum width=3cm, minimum height=1cm, text centered, draw=black, fill=blue!30]
\tikzstyle{process} = [rectangle, minimum width=3cm, minimum height=1cm, text centered, draw=black, fill=orange!30]
\tikzstyle{decision} = [diamond, minimum width=3cm, minimum height=1cm, text centered, draw=black, fill=green!30]
\tikzstyle{arrow} = [thick,->,>=stealth]
%--------------------------------------------------


%--------------------Codeblöcke--------------------
\usepackage{listings} %Für Codeblöcke
\usepackage{color} %Farben für Codeblöcke?
\definecolor{dkgreen}{rgb}{0,0.6,0}
\definecolor{gray}{rgb}{0.5,0.5,0.5}
\definecolor{mauve}{rgb}{0.58,0,0.82}

\lstset{
	language=Java,
	aboveskip=3mm,
	belowskip=3mm,
	showstringspaces=false,
	columns=flexible,
	basicstyle={\small\ttfamily},
	numbers=none,
	numberstyle=\tiny\color{gray},
	keywordstyle=\color{blue},
	commentstyle=\color{dkgreen},
	stringstyle=\color{mauve},
	breaklines=true,
	breakatwhitespace=true,
	tabsize=3
}
%-------------------------------------------------

\usepackage{graphicx} %Package für Bilder
\graphicspath{ {./images/} } %Ordner für Bilder

\sloppy % damit lange Wörter nicht über die Zeile hinausgeschrieben werden.

%--------------------Chapter Heading--------------------
\makeatletter
\def\@makechapterhead#1{%
	\vspace*{-20\p@}%
	{\parindent \z@ \raggedright \normalfont
		\ifnum \c@secnumdepth >\m@ne
		%\huge\bfseries \@chapapp\space \thechapter
		\Huge\bfseries \thechapter.\space%
		%\par\nobreak
		%\vskip 20\p@
		\fi
		\interlinepenalty\@M
		\Huge \bfseries #1\par\nobreak
		\vskip 20\p@
}}
\makeatother
\makeatletter
\def\@makeschapterhead#1{%
	\vspace*{-20\p@}%
	{\parindent \z@ \raggedright \normalfont
		%\huge\bfseries \@chapapp\space \thechapter
		\Huge\bfseries\space%
		%\par\nobreak
		%\vskip 20\p@
		\interlinepenalty\@M
		\Huge \bfseries #1\par\nobreak
		\vskip 20\p@
}}
\makeatother
%-------------------------------------------------------

%\addto\captionsngerman{\renewcommand{\listfigurename}{}}
 % titel von abbildungsverzeichnis weg?

\setlength\parindent{0pt} %Auto Einrücken deaktivieren


%-------------Setup für Inhaltsverzeichnis--------------
\renewcommand{\contentsname}{Inhaltsverzeichnis} %Umbenennung TOC

\usepackage{tocloft} % Formatierung TOX

\setlength{\cftbeforetoctitleskip}{0pt}

\renewcommand{\cfttoctitlefont}{\huge\bfseries}
\renewcommand{\cftloftitlefont}{\huge\bfseries}

\renewcommand\cftchapfont{\Large\bfseries}
\renewcommand\cftchappagefont{\large}

\renewcommand\cftsecfont{\large\bfseries}
\renewcommand\cftsecpagefont{\large}

\renewcommand\cftsubsecfont{\large}
\renewcommand\cftsubsecpagefont{\large}

\renewcommand\cftsubsubsecfont{\normalsize}
\renewcommand\cftsubsubsecpagefont{\normalsize}


\renewcommand\cftchapafterpnum{\par\addvspace{8pt}}
\renewcommand\cftsecafterpnum{\par\addvspace{8pt}}
\renewcommand\cftsubsecafterpnum{\par\addvspace{6pt}}
\renewcommand\cftsubsubsecafterpnum{\par\addvspace{6pt}}
%-------------------------------------------------------


%------------------Setup für LoF/LoT--------------------
\makeatletter
\renewcommand{\@cftmakeloftitle}{}
\renewcommand{\@cftmakelottitle}{}
\makeatother
\setlength{\cftfigindent}{0em} % change indentation of e.g. "Figure 1" within list of figures
\renewcommand\cftfigfont{\large}
\renewcommand\cftfigpagefont{\large}
\setlength{\cfttabindent}{0em} % change indentation of e.g. "Figure 1" within list of figures
\renewcommand\cfttabfont{\large}
\renewcommand\cfttabpagefont{\large}
\setlength{\cftbeforeloftitleskip}{0pt}
\setlength{\cftbeforelottitleskip}{0pt}
%-------------------------------------------------------


%----------------Setup für Verlinkungen-----------------
\usepackage{hyperref}
\hypersetup{
	colorlinks,
	citecolor=black,
	filecolor=black,
	linkcolor=black,
	urlcolor=black
}
%-------------------------------------------------------


%-------------Kopf-/Fußzeile für Titlepage--------------
\fancypagestyle{titlepage}
{
	\fancyhead[L]{\includegraphics[scale=0.09]{firmenlogo}}
	\fancyhead[R]{\includegraphics[scale=0.25]{dhbw}}
	\renewcommand{\headrulewidth}{0pt}
	\fancyfoot[C]{}
}
%-------------------------------------------------------

\begin{document} 
	\begin{titlepage}
		\thispagestyle{titlepage}
		\newcommand\HRule{\rule{\textwidth}{1pt}} %Titellinien

		
		\begin{center}
			
			\vspace*{2cm}
			
			%Title
			\begin{spacing}{2}
				{ \huge \bfseries \MakeUppercase{\meinTitel}}
				%{ \large \bfseries subTitle}\\[0.4cm]
			\end{spacing}
			
			\vspace*{1.5cm}
			
			%Art der Arbeit
			\Large \artDerArbeit
			
			\vspace*{3cm}
			
			%Hochschule
			{\LARGE Studiengang Informatik}\\
			{\LARGE an der Dualen Hochschule}\\
			{\LARGE Baden-Württemberg Stuttgart}\\

			\vspace*{2.5cm}
			
			\Large von \meinName
			
			\vspace*{1.5cm}
			
			\Large Abgabedatum: \abgabeDatum

			\begin{table*}[bp]
				\begin{tabular}{l l l l}
					Kurs: & \meinKurs & Matrikelnummer: & \meineMatrikelNr  \\
					Modul: & \modul & Dozent: & \dozent\\
				\end{tabular}
			\end{table*}
			
			
		\end{center}
		
	\end{titlepage}


%------------------Kopf- und Fußzeile-------------------
\spacing{1.5}

\fancypagestyle{plain}{
	\fancyfoot[L]{\meinName\\
		 \meinKurs}
	\fancyfoot[C]{Seite \thepage\ }% von \pageref{LastPage}}
	\fancyfoot[R]{\abgabeDatum}
}

\pagestyle{plain}
\fancyhead{}


\fancyhead[L]{\includegraphics[scale=0.09]{firmenlogo}}
\fancyhead[C]{\nouppercase\leftmark}
\fancyhead[R]{\includegraphics[scale=0.25]{dhbw}}

\renewcommand{\footrulewidth}{0.4pt} %Linie für Fußzeile


\renewcommand{\sectionmark}[1]{\markboth{#1}{}} 
%-------------------------------------------------------

\pagenumbering{Roman}
\newpage
\chapter*{Abstract}
\addcontentsline{toc}{chapter}{\protect\numberline{}Abstract}
Hier kommt der Abstract.
\newpage

%------------------Inhaltsverzeichnis-------------------

\addcontentsline{toc}{chapter}{\protect\numberline{}Inhaltsverzeichnis}
\tableofcontents
\addtocontents{toc}{}
\thispagestyle{plain}
%-------------------------------------------------------


\newpage
\chapter*{Ehrenwörtliche Erklärung}
\addcontentsline{toc}{chapter}{\protect\numberline{}Ehrenwörtliche Erklärung}

Ich erkläre hiermit ehrenwörtlich, dass ich die vorliegende Arbeit selbstständig und ohne Benutzung anderer als der angegebenen Hilfsmittel angefertigt habe. Aus den benutzten Quellen, direkt oder indirekt, übernommene Gedanken habe ich als solche kenntlich gemacht.
\\
\\
\\
Diese Arbeit wurde bisher in gleicher oder ähnlicher Form oder auszugsweise noch keiner anderen Prüfungsbehörde vorgelegt und auch nicht veröffentlicht

\vspace{2.5cm}

\parbox{6.5cm}{\hrule\medskip Ort, Datum}  \hfill \parbox{6.5cm}{\hrule\medskip Unterschrift} \hspace{3cm}


%********************************
%Abbildungsverzeichnis
%********************************
\newpage
\chapter*{Abbildungsverzeichnis}
\addcontentsline{toc}{chapter}{\protect\numberline{}Abbildungsverzeichnis}

\listoffigures




%********************************
%Tabellenverzeichnis
%********************************
\newpage
\chapter*{Tabellenverzeichnis}
\addcontentsline{toc}{chapter}{\protect\numberline{}Tabellenverzeichnis}

\listoftables


\newpage
\chapter*{Abkürzungsverzeichnis}
\addcontentsline{toc}{chapter}{\protect\numberline{}Abkürzungsverzeichnis}
\markboth{Abkürzungsverzeichnis}{Abkürzungsverzeichnis} %Benötigt, damit das im header steht

%********************************
%Abkürzungsverzeichnis
%********************************
\begin{acronym}[SOAP]
	\acro{BA}{Beispiel Acronym}
\end{acronym}


%Speichern des page counters, um bei Literaturverzeichnis weiter zu zählen.
\newcounter{frontmatterPage}
\addtocounter{frontmatterPage}{\value{page}} 

\newpage
\pagenumbering{arabic}
\chapter{Einleitung}
%********************************
%Einleitung
%********************************




%********************************
%Aufgabenstellung
%********************************
\pagebreak
\section{Problemstellung}


%********************************
%Vorüberlegung
%********************************
\pagebreak
\section{Ziel der Hausarbeit}

%********************************
%Vorüberlegung
%********************************
\pagebreak
\section{Struktur der Hausarbeit}


%********************************
%Lösung
%********************************
\newpage
\chapter{Grundlagen des Rucksackproblems}


%********************************
%Iwas eigenes
%********************************
\section{Definition}
Gegeben sei eine Menge $U$ von Objekten. Jedem Objekt $u \in U$ wird ein Gewicht $w(u)$ und ein Nutzenwert $v(u)$ zugeordnet. Es gibt außerdem eine Gewichtsschranke $B \in \mathbb{R}$.

Gesucht ist eine Teilmenge $K \subseteq U$, die die Bedingung

$$\sum_{u \in K} w(u) \leq B$$

einhält und die Zielfunktion

$$\sum_{u \in K} v(u)$$

maximiert. \cite{kellererknapsack}

%********************************
%Iwas eigenes
%********************************
\section{Anwendungsgebiete}
Das Rucksackproblem ist ein klassisches Optimierungsproblem, das in vielen Bereichen der realen Welt Anwendung findet. Hier sind einige Beispiele:

\begin{itemize}
	\item \textbf{Ressourcenoptimierung:}
	\begin{itemize}
		\item \textbf{Zuschnitt von Rohmaterialien:} Das Rucksackproblem kann verwendet werden, um den Verschnitt beim Zuschnitt von Rohmaterialien zu minimieren. 
		\item \textbf{Auswahl von Investitionen:} Das Rucksackproblem kann verwendet werden, um die optimale Auswahl von Investitionen für ein Portfolio zu finden.
		\item \textbf{Auswahl von Vermögenswerten:} Das Rucksackproblem kann verwendet werden, um die optimale Auswahl von Vermögenswerten für eine Asset-Backed Securitization zu finden.
	\end{itemize}
	\item \textbf{Kryptographie:}
	\begin{itemize}
		\item \textbf{Generierung von Schlüsseln:} Das Rucksackproblem kann verwendet werden, um Schlüssel für kryptographische Systeme wie das Merkle-Hellman-Kryptosystem zu generieren.
	\end{itemize}
	\item \textbf{Prüfungsdesign:}
	\begin{itemize}
		\item \textbf{Zusammenstellung von Prüfungen:} Das Rucksackproblem kann verwendet werden, um Prüfungen so zusammenzustellen, dass die Schüler die höchstmögliche Punktzahl erreichen können.
	\end{itemize}
	\item \textbf{Weitere Anwendungsgebiete:}
	\begin{itemize}
		\item Produktionsplanung
		\item Personalplanung
		\item Logistik
		\item Telekommunikation
	\end{itemize}
\end{itemize}

%********************************
%Iwas eigenes
%********************************
\section{Komplexitätsanalyse}



%********************************
%Lösung
%********************************
\newpage
\chapter{Dynamisches Programmieren}


%********************************
%Iwas eigenes
%********************************
\pagebreak
\section{Prinzip des dynamischen Programmierens}

%********************************
%Iwas eigenes
%********************************
\pagebreak
\section{Implementierung des dynamischen Programmierens für das Rucksackproblem}

%********************************
%Iwas eigenes
%********************************
\pagebreak
\section{Laufzeitanalyse und Komplexität}



%********************************
%Lösung
%********************************
\newpage
\chapter{Greedy-Algorithmen}


%********************************
%Iwas eigenes
%********************************
\pagebreak
\section{Prinzip der Greedy-Algorithmen}

%********************************
%Iwas eigenes
%********************************
\pagebreak
\section{Beispiele für Greedy-Lösungsansätze}

%********************************
%Iwas eigenes
%********************************
\pagebreak
\section{Vergleich mit dynamischem Programmieren}



%********************************
%Lösung
%********************************
\newpage
\chapter{Heuristische Ansätze}


%********************************
%Iwas eigenes
%********************************
\pagebreak
\section{Genetische Algorithmen}

%********************************
%Iwas eigenes
%********************************
\pagebreak
\section{Simulierte Abkühlung (Simulated Annealing)}

%********************************
%Iwas eigenes
%********************************
\pagebreak
\section{Tabu-Suche}

%********************************
%Iwas eigenes
%********************************
\pagebreak
\section{Vergleich der Heuristiken und ihrer Effizienz}



%********************************
%Lösung
%********************************
\newpage
\chapter{Anwendungsbeispiele und Fallstudien}


%********************************
%Iwas eigenes
%********************************
\pagebreak
\section{Praktische Beispiele für das Rucksackproblem}

%********************************
%Iwas eigenes
%********************************
\pagebreak
\section{Fallstudien zur Anwendung von Rucksackproblemlösungen}


%********************************
%Zusammenfassung und Ausblick
%********************************
\newpage
\chapter{Zusammenfassung und Ausblick}

%********************************
%Iwas eigenes
%********************************
\pagebreak
\section{Kritische Reflexion}

%********************************
%Iwas eigenes
%********************************
\pagebreak
\section{Ausblick auf zukünftige Entwicklungen und Forschungsrichtungen}


%********************************
%Literaturverzeichnis
%********************************
\newpage
\pagenumbering{Roman}
\setcounter{page}{\value{frontmatterPage}} %Bei \pagenumbering wird Seitenzähler zurückgesetzt, hier wird der gespeicherte Wert vom frontmatter weitergeführt
\addtocounter{page}{1}
\addcontentsline{toc}{chapter}{\protect\numberline{}Literaturverzeichnis}

%Quellenverzeichnis
\renewcommand{\refname}{Literaturverzeichnis}
\bibliographystyle{IEEEtran}
\bibliography{./sources}


\end{document}
